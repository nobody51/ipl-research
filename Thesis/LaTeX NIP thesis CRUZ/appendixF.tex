\chapter{Steady-state Simulations}
\label{apndx:steadstatesim}
Shown is the code that outputs the simulations with the analytical steady-state as well as the mean of the system for a single set of parameters.

\begin{lstlisting}

#####################################################
#Simulates applause for a specific set of parameters#
#Displays the appropriate steady state value        #
#Parameters can be adjusted accordingly below       #
#####################################################


import applause_functions as app
import matplotlib.pyplot as plt
import numpy as np

N = 10
M = 10
population = N * M
time = 500
t_1 = 2
C = 0
aStoC = 1
bCtoS = 0.8
alpha = 0.6
beta = 1

fig = plt.figure()
ax = fig.add_subplot(111)
sim = app.app_sim(aStoC, bCtoS, alpha, beta, N, M, C, time, t_1)
plt.plot(sim)
ax.text(0.45*time,1,'Steady-state:'+str(round(app.steady_nC(aStoC, bCtoS, alpha, beta, population, 1),0)), fontsize=20)
ax.text(0.45*time,10,'Mean: '+str(round(np.mean(sim),1))+'$\pm$'+str(round(np.std(sim),2)), fontsize=20)
ax.text(10,100,'N = '+str(population), fontsize=15)
plt.axhline(np.mean(sim), color = 'r')
plt.axhline(app.steady_nC(aStoC, bCtoS, alpha, beta, population, 1), color = 'g')
plt.title('$a = $'+str(aStoC)+' $b = $'+str(bCtoS)+r' $\alpha=$'+str(alpha)+r' $\beta=$'+str(beta),fontsize=20)
plt.xlabel('Time',fontsize=18)
plt.ylabel('State'+r' $n_{c}$',fontsize=18)
plt.xlim(0,time)
plt.ylim(0,110)
plt.savefig('ssB.png')
plt.show()

\end{lstlisting}

Following is the code used to generate the data points of the simulated steady-state values to be plotted against the analytic steady-state solutions.

\begin{lstlisting}

#Takes the last point of the simulation (Trials) times then calculates mean and std#
#Does so for varying a-bar values                                                  #
#Creates .csv file with columns: a-bar, mean, std                                  #
#Plots the data points                                                             #
#Parameters can be adjusted accordingly below                                      #


import applause_functions as app
import matplotlib.pyplot as plt
import numpy as np

#####Simulation Parameters#####
N = 10
M = 10
startPop = 20
time = 100
t_1 = 0
aStoC = 1.0
bCtoS = 0.8 
alpha = 1
beta = 10
###Point Generator Parameters###
Trials = 50
interval = 0.02
##############################
def pointGen(aStoC, bCtoS, beta, N, M, startPop, time, t_1, Trials, interval):
    
    meanlist = []
    stdlist = []
    ilist = []
    for i in np.arange(0,1.0,interval):
        list_a = []
        for j in range(Trials):
            list_a.append(app.app_sim(aStoC, bCtoS, i, beta, N, M, startPop, time, t_1)[time-1]) #takes the last number from the appsim vector
        mean = np.mean(np.array(list_a))
        std = np.std(np.array(list_a))  
        meanlist.append(mean)
        stdlist.append(std)
        ilist.append(i)

    ilist = np.array(ilist)
    meanlist = np.array(meanlist)
    stdlist = np.array(stdlist)
    
    #writes ilist,meanlist, and stdlist into a txt file
    filename = 'b = ' + str(bCtoS) + ', beta = ' + str(beta) + ', startPop = ' + str(startPop) + '.txt'    
    with open(filename,'w') as f:
        lis=[ilist,meanlist,stdlist]
        for x in zip(*lis):
            f.write("{0}\t{1}\t{2}\n".format(*x))
            
    return (ilist,meanlist,stdlist)
\end{lstlisting}