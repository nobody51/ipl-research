The audience applause is a social phenomena that is very rich in dynamics. 
This human behavior is a collective of interacting agents that can be treated as a complex system.
Studies have treated the audience applause similarly to a disease in order to apply epidemic models to analyze the dynamics.
A compartmental SCS model based on the SIS epidemic model is created and simulated using the Monte Carlo method where state transitions $\mathrm{S} \rightarrow \mathrm{C} \rightarrow \mathrm{S}$.
Functions are included in order to reproduce the audience applause as accurately as possible.
$f$ is a forcing function that initiates the state transitions.
$f'(\alpha)$ is a feedback function that promotes agents to transition to state C if more agents are in state C.
$g'(\beta)$ is a modulation function that inhibits agents in state C to transition to state S.
Simulations with specific parameter sets $(a,b,\alpha,\beta)$ show cases where the state $n_{c}$ never reaches zero, or that the applause never ends (which doesn't occur in real life).
The steady-state of the system is calculated and plotted in phase space in order to analyze its dynamics.
For $\beta \leq 1$, there are two solutions, the trivial ($n^\infty_{c}=0$), and non-trivial steady-state ($n^\infty_{c}>0$).
For $\beta > 1$, there are three solutions, the trivial solution, and two non-trivial solutions.
These two solutions are the upper branch ($n^\infty_{c1}>0$) and the lower branch of the curve ($n^\infty_{c2}>0$).
Critical points have been observed to which the system experiences bifurcation from the trivial steady-state to the non-trivial steady-state.
For parameter sets with $\beta >1$, the lower branch was shown to be unstable.
If the system starts in state $n_{c}$ with $\bar{a}$ values on the lower branch of the curve, it can settle to either trivial or non-trivial solution.
The original SCS model exhibited scale-free behavior when the population size was increased.
Spatial effects were incorporated to the SCS model in order to more appropriately model the audience applause.
This is then confirmed by comparing with a real-life applause data set.








