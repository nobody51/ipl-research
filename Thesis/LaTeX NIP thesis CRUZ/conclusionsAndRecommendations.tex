\chapter{Conclusions and Recommendations}
\label{conclusions}

\hspace{\parindent} A compartmental model SIS-like model was successfully constructed to study the complexity of the audience applause, as well as simulate it with an agent-based Monte Carlo method. 
The original, fully-connected system exhibited parameter sets that express a  non-trivial steady-state solution. The dynamics of the given system were studied, uncovering critical points in the phase space plot. 
These critical points provide the point at which the model bifurcates from the trivial steady-state solution to the non-trivial stready-state solution.
For $\beta \leq 1$, the critical point is $\bar{a}_{1} \approx b$.
The system follows the trivial solution for $\bar{a} \leq \bar{a}_{1}$ and then bifurcates at the critical point.
After which, it follows the non-trivial solution.
For $\beta > 1$, there is a second critical point $\bar{a}_2 = \frac{b(N-1)^{2}}{(N-n_{c}^{*})(N-1+\beta n_{c}^{*})}$ where $n_{c}^{*} \equiv [1+(\beta-1)N]/2\beta$.
The system follows the trivial solution for $\bar{a} \leq \bar{a}_{2}$ and then bifurcates somewhere before critical point.
Bifurcation for parameter sets with $\beta > 1$ are only partially predicted.
Once it bifurcates to the non-trivial steady-state, it follows the upper branch of the curve.
Also, the lower branch of $\beta >1$ curves are shown to be unstable; if the system starts with a state $n_{c}$ on the lower branch, it can either settle to the trivial or non-trivial solution.
Further modifications were made to the compartmental model in order to incorporate spatial effects, as such effects more closely resembled a real-life applause.
Three spatial configurations were investigated, $\theta = 0, \frac{\pi}{2}, \pi$, each $\theta$ value corresponding to the view of the reference agent.
Each configuration successfully incorporated the population size dependence of the applause duration.
$\theta = \pi$ was used as it was the most realistic as well as the easiest to implement in code.
This allowed simulations to show a correlation between applause duration and audience size.
The simulations were backed by data points of real-life applause.
Analysis of the model with incorporated spatial effects is necessary in order to understand the dynamics of the audience applause.
This will allow further development of the model, which in turn, will more closely resemble the real-life applause.

