\chapter{Conclusions and Recommendations}
\label{conclusions}

\hspace{\parindent} A compartmental model SIS-like model was successfully constructed to study the complexity of the audience applause, as well as simulate it with an agent-based Monte Carlo method. 
The original, fully-connected system exhibited parameter sets that express a  non-trivial steady-state solution. The dynamics of the given system were studied, uncovering critical points in the phase space plot, as well as unstable points.
Further modifications were made to the compartmental model in order to incorporate spatial effects, as such effects more closely resembled a real-life applause.
Doing so removed the scale-free property of the original fully-connected system.
This allowed simulations to show a correlation between applause duration and audience size.
The simulations were backed by data points of real-life applause.
Analysis of the model with incorporated spatial effects is necessary in order to understand the dynamics of the audience applause.
This will allow further development of the model, which in turn, will more closely resemble the real-life applause.

