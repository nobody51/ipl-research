\chapter{Monte Carlo Compartmental Method}
\label{chap2}
\def\a{(n \pi p)}
\def\aa{(\bar{n} \pi p)}
\def\b{kpL}
\def\c{k(1-p)L}
\def\s{\\sin{(\bar{n}\pi p)}}
\def\c{\\cos{(\bar{n}\pi p)}}
\def\statepsi{\mid \psi \; \rangle}
\def\energy{\mid E_{\vec{k}} \; \rangle}
\def\psixt{\mid \psi(x,t) \; \rangle}
\def\statepsixtrev{\mid \psi(x,t=T_{rev}) \; \rangle}
\def\statepsixt0{\mid \psi(x,t=0) \; \rangle}
\def\lowering{S^-_l \mid 0 \; \rangle}
\def\loweringa{S^-_m \; S^-_l \mid 0 \; \rangle}

%\hspace{\parindent} this forces the indentation of the first paragraph.

\section{The applause as a contagion}
\hspace{\parindent} Previous studies have treated the applause as a contagion that spreads across the audience. Doing this allows the use of an SIR compartmental model to simulate the complex system.
A more appropriate model would be an SIS model since this would account for the people who stop clapping but may do so again. 

\section{Compartmental Model}
\subsection{States}
An SIS-like model is proposed where each agent in the system is either in state S (silent) or state C (clapping). State S replaces the susceptible state while state C replaces the infected state.
The state of the system is given by the number of agents in each corresponding state, $\vec{n}\equiv(n_{c},n_{s})$.
Since we assume that the total number of agents is fixed at $N = n_{c} + n_{s}$ such that $n_{s} = N-n_{c}$, $\vec{n}$ will be fully specified by $n_{c}$ alone.

\subsection{Parameters}
The parameters $a$ and $b$ are the transition probabilities for the respective transitions:
\begin{eqnarray}
\mathrm{R}_{1} &:& \mathrm{S} \overset{a}{\longrightarrow} \mathrm{C} \label{eq:r1} \\
\mathrm{R}_{2} &:& \mathrm{C} \overset{b}{\longrightarrow} \mathrm{S}.\label{eq:r2}
\end{eqnarray}
The parameter $f$ is a function that forces the transition $\mathrm{R}_{1}$ for an indicated time interval such that when $f=0$, the system behaves freely.
We assume that agents in state S may be encouraged by the agents already in state C to spontaneously undergo $\mathrm{R}_{1}$.

\subsection{Functions}
The function $f'$ incorporates this feedback mechanism parametrized by $\alpha$:
\begin{equation}\label{eq:f'}
  f'(\alpha) = \alpha \frac{n_c}{N-1},
\end{equation}
where $0 \leq \alpha \leq 1$ for consistency with probabilistic interpretations.
The probability for a spontaneous ${R}_{1}$ transition is directly proportional to around a fraction of the population in state C and $\alpha$. The denominator is set to $N-1$ because an agent cannot spontaneously influence itself; it is only influenced by the rest of the population.
The function is more effective when there are more agents in state C.

Another assumption is that the presence of a large applauding audience 
%results to a kind of inhibition of 
inhibits those already clapping to undergo $\mathrm{R}_{2}$.
The factor $g'$ incorporates a modulation function for this inhibition and is parametrized by $\beta$: 
%%
\begin{equation}\label{eq:g'}
  g'(\beta) = \frac{1}{1 + \beta\;n_C /(N-1)}
\end{equation}
where $\beta \geq 0$. This equation is taken from the Michaelis-Menten equation, which aims to model enzyme kinetics\cite{michaelisconstant}.
This completes the differential equations for the reactions \eqref{eq:r1} and \eqref{eq:r2} as follows.
\begin{eqnarray}
\frac{d}{dt}n_{c} &=& a (f+f'-f'f) n_{s} - b g' n_{c}\label{eq:diff1} \\
\frac{d}{dt}n_{s} &=& b g' n_{c} - a (f+f'-f'f) n_{s}\label{eq:diff2}
\end{eqnarray}
These equations are consistent with the assumption that the total audience size is fixed, that is $dn_{c}/dt = -dn_{s}/dt$.

\section{Simulation Alogrithm}


